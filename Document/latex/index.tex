\section*{基于\+M\+F\+C的文本编辑器 }

\subsection*{目标 }


\begin{DoxyEnumerate}
\item 不使用\+M\+F\+C文本框控件,实现一个简单的文本编辑器
\item 最低要求如下: 带有图形界面 能够比较方便的给不同的字符设置不同的字体、颜色和字号 可以设置字与字之间的间距
\item 另外还实现了以下功能 保存、打开自定义格式的文件 设置对齐方式,包括左对齐、居中对齐、右对齐、分散对齐 设置行间距
\end{DoxyEnumerate}

\subsection*{使用方法 }


\begin{DoxyEnumerate}
\item 常规操作与普通文本编辑器相同
\item 设置字体、颜色、字号:
\begin{DoxyItemize}
\item 选中想要改变的文本内容
\item 点击编辑菜单栏,点击字体,将会弹出选择字体的窗口
\item 选择想要设置的格式即可
\end{DoxyItemize}
\item 设置对齐方式:
\begin{DoxyItemize}
\item 如果只需要设置单行的格式,请直接将光标移到该行
\item 如果需要设置多行的格式,请选中多行
\item 点击编辑菜单栏,点击对齐方式
\item 在弹出的右侧菜单栏中选择需要的对齐方式
\end{DoxyItemize}
\item 设置行间距、字间距:
\begin{DoxyItemize}
\item 选中需要设置的文本内容
\item 点击编辑菜单栏,点击段落
\item 在弹出的窗口中输入需要设置的行间距以及字间距
\item 注意单位为像素
\end{DoxyItemize}
\item 打开与保存
\begin{DoxyItemize}
\item 首先注意,本文本编辑器只支持自定义格式的(后缀名为$<$tt$>$.\+orz)文件,其他格式的文件一律无法打开
\item 打开:点击菜单栏,点击打开,在文件浏览器中选择想要打开的文件,点击确定即可
\item 保存:点击菜单栏,点击保存,在文件浏览器中选择想要保存的目录,点击确定即可
\end{DoxyItemize}
\end{DoxyEnumerate}

\subsection*{构件库 }

详情请参见\+Class List

\subsection*{U\+I界面的交互实现方法说明 }


\begin{DoxyEnumerate}
\item 设置字体字号颜色 ~\newline
通过{\ttfamily C\+Font\+Dialog}窗口类,创建选择字体的窗口 ~\newline
用户选择字体以及颜色后,获取字体信息{\ttfamily L\+O\+G\+F\+O\+NT}以及颜色信息{\ttfamily R\+E\+F\+C\+O\+L\+OR}~\newline
通过{\ttfamily set\+\_\+select\+\_\+font}与{\ttfamily set\+\_\+select\+\_\+col}函数将字体以及颜色信息传递给后端~\newline

\item 设置字间距行间距 ~\newline
通过{\ttfamily \hyperlink{class_m___p_a_r_a___d_i_a}{M\+\_\+\+P\+A\+R\+A\+\_\+\+D\+IA}}对话框,获取用户设置的字间距与行间距的值 ~\newline
通过{\ttfamily set\+\_\+select\+\_\+lspace}与{\ttfamily set\+\_\+select\+\_\+cspace}函数将字间距与行间距传给后端~\newline

\item 保存、打开文件 ~\newline
通过{\ttfamily C\+File\+Dialog}类,返回用户需要保存或者打开文件的路径~\newline
通过{\ttfamily \#}与{\ttfamily \#}函数将路径作为字符串传给后端~\newline

\item 设置对齐方式 ~\newline
直接通过消息响应函数,调用{\ttfamily set\+\_\+select\+\_\+align}函数,设置对齐方式~\newline

\end{DoxyEnumerate}

\subsection*{后端(kernal)实现方法说明 }


\begin{DoxyEnumerate}
\item 抽象~\newline

\begin{DoxyItemize}
\item 把画在屏幕上的字符看作一个矩形~\newline

\item 用链表从前往后储存字符~\newline

\end{DoxyItemize}
\item 计算绘制信息~\newline

\begin{DoxyItemize}
\item 遍历文本链表,进行预处理~\newline

\item 遍历文本链表,把链表分成很多行~\newline

\begin{DoxyItemize}
\item 当某行的字符宽度之和({\ttfamily tot\+\_\+width})将要超过页面宽度({\ttfamily pagewidth})时,开一个新行~\newline

\item 当遇到换行符时,开一格新行~\newline

\end{DoxyItemize}
\item 从上往下遍历每一行,根据相关信息计算该行内的{\ttfamily n}个字符矩形的位置~\newline

\begin{DoxyItemize}
\item 如果该行是默认对齐方式或者左对齐,那么x坐标从0开始,每个矩形宽度加上字间距,从左往右计算~\newline

\item 如果该行是右对齐,那么x坐标从{\ttfamily pagewidth}-\/{\ttfamily tot\+\_\+width}开始,每个矩形宽度加上字间距,从左往右计算~\newline

\item 如果对齐方式是居中对齐,那么x坐标从({\ttfamily pagewidth}-\/{\ttfamily tot\+\_\+width})/2开始,每个矩形宽度加上字间距,从左往右计算~\newline

\item 如果对齐方式是分散对齐,那么x坐标从0开始,每个矩形宽度加上字间距和{\ttfamily delta}=({\ttfamily pagewidth}-\/{\ttfamily tot\+\_\+width})/{\ttfamily n}~\newline

\end{DoxyItemize}
\end{DoxyItemize}
\end{DoxyEnumerate}

\subsection*{U\+I及交互部分遇到的问题及解决方案 }


\begin{DoxyEnumerate}
\item 绘图过程闪烁的问题 ~\newline
问题描述:使用过程中每当需要屏幕刷新的时候,画面会不停的闪烁 ~\newline
可能原因:每次调用\+On\+Paint函数的时候都会先将屏幕清空,由于绘制屏幕仍然需要一定的时间,所以每次重绘都会出现闪烁的情况 ~\newline
解决方案:使用双缓冲绘图,分为两个部分:~\newline
(1)重载{\ttfamily B\+O\+OL \hyperlink{class_c_child_view_a6060e6d09d522d345dcee5a01d41c1f0}{C\+Child\+View\+::\+On\+Erase\+Bkgnd(\+C\+D\+C$\ast$ p\+D\+C)}}函数,使得画面不会被擦除~\newline
(2)先在使用内存\+D\+C上画图,画好后直接调用{\ttfamily Bit\+Blt}函数将画面传给物理\+DC~\newline

\item 滚动条位置以及大小的问题~\newline
问题描述:滚动条大小以及位置随着窗口大小改变而随机变动~\newline
可能原因:由于滚动条控件需要手动设置大小以及位置,所以当改变窗口大小的时候会出现滚动条位置偏移的情况~\newline
解决方案:设置两个变量:{\ttfamily m\+\_\+client\+\_\+cy}记录页面长度,{\ttfamily scrolledpix}记录页面相对于窗口上端的偏移量~\newline
每次重绘的时候通过这两个量来重新计算滚动条的大小以及位置~\newline

\item 解析用户传入的键盘消息~\newline
问题描述:根据官方文档,键盘消息的对应关系很复杂,需要解析传入的参数才能获得字符信息~\newline
解决方案:使用{\ttfamily void \hyperlink{class_c_child_view_af29ede94259b52b2ad54d139ff554abe}{C\+Child\+View\+::\+On\+Char(\+U\+I\+N\+T n\+Char, U\+I\+N\+T n\+Rep\+Cnt, U\+I\+N\+T n\+Flags)}}的消息响应函数~\newline
系统将键盘消息处理后返回用户输入的字符~\newline

\item 连按消息的问题~\newline
问题描述:通常连按一个按键,被认为是连续输入字符~\newline
解决方案:在{\ttfamily void \hyperlink{class_c_child_view_a74d87512b76128e2eedea87811363e45}{C\+Child\+View\+::\+On\+Key\+Down(\+U\+I\+N\+T n\+Char, U\+I\+N\+T n\+Rep\+Cnt, U\+I\+N\+T n\+Flags)}}函数中~\newline
{\ttfamily n\+Rep\+Cnt}参数是用于标记连按消息的次数的,循环相应的次数即可实现连续输入~\newline

\item 获取字体的像素大小的问题~\newline
问题描述:{\ttfamily L\+O\+G\+F\+O\+NT}结构体的成员{\ttfamily lf\+Height}与{\ttfamily lf\+Width}不是像素高度~\newline
解决方案:通过公式{\ttfamily pix\+Height = -\/(lf\+Height $\ast$ 72) / Get\+Device\+Caps(h\+D\+C,\+L\+O\+G\+P\+I\+X\+E\+L\+S\+Y)}转换为像素高度~\newline
参考资料:\href{https://msdn.microsoft.com/zh-cn/library/bb773327.aspx}{\tt M\+S\+D\+N上关于\+L\+O\+G\+F\+O\+N\+T的说明}~\newline

\end{DoxyEnumerate}

\subsection*{后端(kernal)遇到的问题及解决方案 }


\begin{DoxyEnumerate}
\item 空行的问题~\newline
问题描述:链表中换行符只用作开启一行的标识符,如果用户连续键入两个换行,第二个换行会变成\char`\"{}空行\char`\"{}而无法在屏幕上打出来~\newline
解决方案:在预处理中,先遍历链表,在相邻两个换行符之间插入一个{\ttfamily S.\+width}=0,{\ttfamily size}=-\/1的字符,避免\char`\"{}空行\char`\"{}的产生~\newline

\item 使用\char`\"{}打开\char`\"{}命令后单击屏幕会崩溃的问题~\newline
问题描述:添加\char`\"{}打开\char`\"{}模块后,用户在界面上单击会导致崩溃~\newline
问题原因:执行\char`\"{}打开\char`\"{}命令后,后端的选中信息会丢失,而\+U\+I当作用户\char`\"{}完成\char`\"{}了一次选中操作调用了{\ttfamily end\+\_\+select}函数,导致内存无效访问~\newline
解决方案:在对选中区域执行任何操作前,先检查是否存在选中区域~\newline

\item \char`\"{}打开\char`\"{}命令对于某些字体无法执行的问题~\newline
问题描述:当打开包含字体名称({\ttfamily lf\+Face\+Name})为多个单词的文件时,\char`\"{}打开\char`\"{}操作会失败~\newline
问题原因:字体名称是{\ttfamily wchar\+\_\+t}类型的,最初我用的scanf(\char`\"{}\%ls\char`\"{})来读取字体名称,这样就只能读到字体名称的第一个单词,而且我没有找到~\newline
针对{\ttfamily wchar\+\_\+t}数组的getline函数。 解决方案:手写针对{\ttfamily wchar\+\_\+t}的{\ttfamily si\+\_\+getline}函数~\newline
 \subsection*{测试用例 }
\end{DoxyEnumerate}


\begin{DoxyEnumerate}
\item {\ttfamily align.\+orz}:测试四种对齐方式
\item {\ttfamily keepcalm.\+orz}:测试设置字体、字号、颜色
\item {\ttfamily linechaspace.\+orz}:测试设置行间距、字间距
\end{DoxyEnumerate}

\subsection*{总结 }


\begin{DoxyEnumerate}
\item 程序具有一定的可使用性,实现了基本目标,在数据规模不太大的情况下可以达到较好效果
\item 当数据规模较大时(超过50000字符),会出现明显的卡顿现象,所以程序仍然有很大改进空间 可以通过每次只计算、绘制当前屏幕附近的字符来解决这个问题
\end{DoxyEnumerate}

\subsection*{作者信息 }


\begin{DoxyItemize}
\item kernal部分作者:李仁杰~\newline

\begin{DoxyItemize}
\item 学院:清华大学软件学院~\newline

\item 班级:软件62~\newline

\item 学号:2016013271~\newline

\item e-\/mail:\+Shadow\+Iterator@hotmail.\+com~\newline

\item 电话:18801005736~\newline

\end{DoxyItemize}
\item U\+I及交互部分作者:洪方舟~\newline

\begin{DoxyItemize}
\item 学院:清华大学软件学院~\newline

\item 班级:软件62~\newline

\item 学号: 2016013259~\newline

\item e-\/mail\+: \href{mailto:hongfz16@163.com}{\tt hongfz16@163.\+com}~\newline

\item 电话:15062727188~\newline

\end{DoxyItemize}
\end{DoxyItemize}